%no borrar PREAMBULO
\documentclass[12pt]{article}

\usepackage[top=3.5 cm, bottom=2.5  cm, left=3 cm, right=3 cm]{geometry}
\usepackage{fancyhdr}
\pagestyle{fancy}

\usepackage[hidelinks]{hyperref} %esta opción saca las cajas de colores de los hiperlinks

\fancyfoot[C]{\thepage }  %numera las páginas

\usepackage[utf8]{inputenc}

\usepackage{amsmath,amsfonts,amssymb}
\usepackage{xcolor}
\usepackage{fancyvrb}
\newcommand\verbbf[1]{\textcolor[rgb]{0,0,1}}%comando para colorear el texto en verbatim

%\linespread{1} %por si queremos achicar el espacio entre lineas

\usepackage{tabularx,booktabs}
\usepackage{graphicx}
\usepackage{float} %para que las figuras puedan ponerse en cualquier lado

\usepackage{subcaption}
\usepackage{layout}
\usepackage{multicol}  %para escribir en columnnas 
\usepackage{float}
\usepackage{textcomp}
\usepackage{natbib}
\usepackage{tikz}
\usepackage{multirow} %para cambiar el alto de una fila en una tabla
\tikzset{
  connect/.style = { dashed, gray }
}
\usepackage{pgfplots}
\pgfplotsset{compat=1.8}
\usepackage[english ,spanish]{babel}
\usepackage{latexsym}
\usepackage{verbatim}

%\usepackage{alltt}
\usepackage{indentfirst}

\usepackage{fancybox, calc} 

\usepackage[flushmargin]{footmisc} %para alinear las notas de página

\usepackage{url}
\usepackage{advdate}
\usepackage{wrapfig}
\usepackage{amsthm}
\usepackage[inline]{enumitem} %para hacer listas en una linea, los mismos comandos con *
\newtheorem*{myteo}{Teorema} % la * es para no numerarlos
\newtheorem*{myexample}{Ejemplo}
\newtheorem*{myprop}{Proposición}
\newtheorem*{mylem}{Lema}
\theoremstyle{definition}
\newtheorem*{mydef}{Definición}
\newtheorem{ejer}{Ejercicio}
\newtheorem*{mydefs}{Definiciones}
%\theoremstyle{remark}
\newtheorem*{myobs}{Observación}
\newtheorem*{remark}{Importante}

\renewcommand{\baselinestretch}{1}  %interlineado

\addto\captionsspanish{%
  \renewcommand{\figurename}{Figura}%
}

\newcommand\myText[1]{\text{\scriptsize\tabular[t]{@{}l@{}}#1\endtabular}}
\addto\captionsspanish{%
  \renewcommand{\tablename}{Tabla}%
}

\def \ds {\displaystyle} %define un comando abreviado  
\def\com{“R”}

\usepackage{hyperref}%para referencias de internet con link!
\newcommand*{\fullref}[1]{\hyperref[{#1}]{ \nameref*{#1}}}
%comando \fullref para que ademas del número de capitulo, sección etc. escriba el título del capitulo, sección o lo que sea a lo que estamos haciendo referencia

\newcommand\comentario[1]{\textcolor{red}{#1}}%comentarios en el pdf

\interfootnotelinepenalty=10000 %previene que se pasen a otra página las notas de pie
\raggedbottom 
\addtolength{\topskip}{0pt plus 10pt}
\addtolength{\footnotesep}{0.1mm}

\VerbatimFootnotes%para poder usar Verbatim en las notas de pie

\begin{document}

\begin{enumerate}

\item  
%q
Dada la sucesión definida por:
\begin{equation*}
a(n) =
\begin{cases} 
  \frac{-n^2+3n-6}{4n^2-2n}  & \text{si  n es impar} \\
..............& \text{si  n es par}
\end{cases}
\end{equation*}

en caso de ser posible, completarla de modo que sea

 \begin{enumerate}
        \item convergente
        \item divergente
        \item ni convergente ni divergente
\end{enumerate}
En cualquier caso, justificar la respuesta.

\item  
%q
Dada la sucesión definida por:
\begin{equation*}
	a(n) =
	\begin{cases} 
		e^{\frac{-n+3}{4n^2-1}}  & \text{si  n es impar} \\
		..............& \text{si  n es par}
	\end{cases}
\end{equation*}

en caso de ser posible, completarla de modo que sea

\begin{enumerate}
	\item convergente
	\item divergente
	\item ni convergente ni divergente
\end{enumerate}
En cualquier caso, justificar la respuesta.

\item  
%q
Dada la sucesión definida por:
\begin{equation*}
	a(n) =
	\begin{cases} 
		e^{\frac{n^2+3}{4n^2-1}}  & \text{si  n es impar} \\
		..............& \text{si  n es par}
	\end{cases}
\end{equation*}

en caso de ser posible, completarla de modo que sea

\begin{enumerate}
	\item convergente
	\item divergente
	\item ni convergente ni divergente
\end{enumerate}
En cualquier caso, justificar la respuesta.

\item  
%q
Dada la sucesión definida por:
\begin{equation*}
	a(n) =
	\begin{cases} 
		e^{\frac{-n^2+3}{4n-1}}  & \text{si  n es impar} \\
		..............& \text{si  n es par}
	\end{cases}
\end{equation*}

en caso de ser posible, completarla de modo que sea

\begin{enumerate}
	\item convergente
	\item divergente
	\item ni convergente ni divergente
\end{enumerate}
En cualquier caso, justificar la respuesta.
%q
\end{enumerate}

\end{document}
