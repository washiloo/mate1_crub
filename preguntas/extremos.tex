%no borrar PREAMBULO
\documentclass[12pt]{article}

\usepackage[top=3.5 cm, bottom=2.5  cm, left=3 cm, right=3 cm]{geometry}
\usepackage{fancyhdr}
\pagestyle{fancy}

\usepackage[hidelinks]{hyperref} %esta opción saca las cajas de colores de los hiperlinks

\fancyfoot[C]{\thepage }  %numera las páginas

\usepackage[utf8]{inputenc}

\usepackage{amsmath,amsfonts,amssymb}
\usepackage{xcolor}
\usepackage{fancyvrb}
\newcommand\verbbf[1]{\textcolor[rgb]{0,0,1}}%comando para colorear el texto en verbatim

%\linespread{1} %por si queremos achicar el espacio entre lineas

\usepackage{tabularx,booktabs}
\usepackage{graphicx}
\usepackage{float} %para que las figuras puedan ponerse en cualquier lado

\usepackage{subcaption}
\usepackage{layout}
\usepackage{multicol}  %para escribir en columnnas 
\usepackage{float}
\usepackage{textcomp}
\usepackage{natbib}
\usepackage{tikz}
\usepackage{multirow} %para cambiar el alto de una fila en una tabla
\tikzset{
  connect/.style = { dashed, gray }
}
\usepackage{pgfplots}
\pgfplotsset{compat=1.8}
\usepackage[english ,spanish]{babel}
\usepackage{latexsym}
\usepackage{verbatim}

%\usepackage{alltt}
\usepackage{indentfirst}

\usepackage{fancybox, calc} 

\usepackage[flushmargin]{footmisc} %para alinear las notas de página

\usepackage{url}
\usepackage{advdate}
\usepackage{wrapfig}
\usepackage{amsthm}
\usepackage[inline]{enumitem} %para hacer listas en una linea, los mismos comandos con *
\newtheorem*{myteo}{Teorema} % la * es para no numerarlos
\newtheorem*{myexample}{Ejemplo}
\newtheorem*{myprop}{Proposición}
\newtheorem*{mylem}{Lema}
\theoremstyle{definition}
\newtheorem*{mydef}{Definición}
\newtheorem{ejer}{Ejercicio}
\newtheorem*{mydefs}{Definiciones}
%\theoremstyle{remark}
\newtheorem*{myobs}{Observación}
\newtheorem*{remark}{Importante}

\renewcommand{\baselinestretch}{1}  %interlineado

\addto\captionsspanish{%
  \renewcommand{\figurename}{Figura}%
}

\newcommand\myText[1]{\text{\scriptsize\tabular[t]{@{}l@{}}#1\endtabular}}
\addto\captionsspanish{%
  \renewcommand{\tablename}{Tabla}%
}

\def \ds {\displaystyle} %define un comando abreviado  
\def\com{“R”}

\usepackage{hyperref}%para referencias de internet con link!
\newcommand*{\fullref}[1]{\hyperref[{#1}]{ \nameref*{#1}}}
%comando \fullref para que ademas del número de capitulo, sección etc. escriba el título del capitulo, sección o lo que sea a lo que estamos haciendo referencia

\newcommand\comentario[1]{\textcolor{red}{#1}}%comentarios en el pdf

\interfootnotelinepenalty=10000 %previene que se pasen a otra página las notas de pie
\raggedbottom 
\addtolength{\topskip}{0pt plus 10pt}
\addtolength{\footnotesep}{0.1mm}

\VerbatimFootnotes%para poder usar Verbatim en las notas de pie

\begin{document}

\begin{enumerate}

\item  
%q
Hallar el valor de los parámetros $a$, $b$ y $c$ para que la función $f(x) = ax^3+x^2+bx+c$ tenga un máximo relativo en $(0,3)$ y un punto de inflexión en $x=1$.

\item  
%q
Sea $f(x)= x^3 + ax^2+bx+5$. Halla $a$ y $b$ para que la curva $y=f(x)$ tenga en $x=1$ un punto de inflexión con tangente horizontal.

\item  
%q
Sea $f(x) = x^2-e^{-ax}$. Calcular el valor de $a$ para que la función tenga un extremo relativo en $x = 2$. Decir de qué tipo de extremo se trata. Explicar.

\item  
%q
Sea la función definida a trozos:
\begin{equation*}
f(x) = 
\begin{cases} 
\frac{1-x}{e^x}& \text{si  } x < 0 \\
 x^2+ax+b& \text{si  } x \geq 0
\end{cases} 
\end{equation*}
\begin{enumerate}
\item Calcular los valores de los parámetros $a$ y $b$ para que la función sea derivable en $x=0$.
\item Encontrar y clasificar los extremos de esta función.
\end{enumerate}

\item  
%q
Sea la función definida a trozos:
\begin{equation*}
f(x) = 
\begin{cases} 
a x^2+bx+ c& \text{si  } x \leq 0 \\
x.\ln{x}& \text{si  } x > 0
\end{cases} 
\end{equation*}
\begin{enumerate}
\item Determinar $a$, $b$ y $c$ para que la función sea continua en $x = 0$, tenga un máximo en $x = -1$ y la tangente en $x = -2$ sea paralela a la recta $y = 2x$. Justificar.
\item ¿Es derivable la función en $x = 0$?. Justificar.
\end{enumerate}

\item  
%q
El número de personas ingresadas en un hospital por una infección después de t semanas viene dado por la función:
\begin{equation*}
N(t) = \frac{350t}{2t^2-3t+8}
\end{equation*}
\begin{enumerate}
\item Calcular el máximo de personas ingresadas y decir en que momento ($t$) ocurre. Justificar.
\item ¿A partir de qué semana, después de alcanzar el máximo, el número de ingresados es menor a $25$ personas?. Justificar.
\end{enumerate}
 
\item  
%q
La función $f(x) = \frac{\ln x}{x^2}$ alcanza extremos locales en:
\begin{enumerate}
\item En ningún punto de su dominio.
\item Sólo en $x = \sqrt{e}$, donde alcanza un mínimo.
\item Sólo en $x = \sqrt{e}$, donde alcanza un máximo.
\item En $x = 0$ alcanza un mínimo y en $x = \sqrt{e}$ alcanza un mínimo.
\end{enumerate}
Justificar.

%q
\end{enumerate}

\end{document}
