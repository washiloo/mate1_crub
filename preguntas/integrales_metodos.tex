%no borrar PREAMBULO
\documentclass[12pt]{article}

\usepackage[top=3.5 cm, bottom=2.5  cm, left=3 cm, right=3 cm]{geometry}
\usepackage{fancyhdr}
\pagestyle{fancy}

\usepackage[hidelinks]{hyperref} %esta opción saca las cajas de colores de los hiperlinks

\fancyfoot[C]{\thepage }  %numera las páginas

\usepackage[utf8]{inputenc}

\usepackage{amsmath,amsfonts,amssymb}
\usepackage{xcolor}
\usepackage{fancyvrb}
\newcommand\verbbf[1]{\textcolor[rgb]{0,0,1}}%comando para colorear el texto en verbatim

%\linespread{1} %por si queremos achicar el espacio entre lineas

\usepackage{tabularx,booktabs}
\usepackage{graphicx}
\usepackage{float} %para que las figuras puedan ponerse en cualquier lado

\usepackage{subcaption}
\usepackage{layout}
\usepackage{multicol}  %para escribir en columnnas 
\usepackage{float}
\usepackage{textcomp}
\usepackage{natbib}
\usepackage{tikz}
\usepackage{multirow} %para cambiar el alto de una fila en una tabla
\tikzset{
  connect/.style = { dashed, gray }
}
\usepackage{pgfplots}
\pgfplotsset{compat=1.8}
\usepackage[english ,spanish]{babel}
\usepackage{latexsym}
\usepackage{verbatim}

%\usepackage{alltt}
\usepackage{indentfirst}

\usepackage{fancybox, calc} 

\usepackage[flushmargin]{footmisc} %para alinear las notas de página

\usepackage{url}
\usepackage{advdate}
\usepackage{wrapfig}
\usepackage{amsthm}
\usepackage[inline]{enumitem} %para hacer listas en una linea, los mismos comandos con *
\newtheorem*{myteo}{Teorema} % la * es para no numerarlos
\newtheorem*{myexample}{Ejemplo}
\newtheorem*{myprop}{Proposición}
\newtheorem*{mylem}{Lema}
\theoremstyle{definition}
\newtheorem*{mydef}{Definición}
\newtheorem{ejer}{Ejercicio}
\newtheorem*{mydefs}{Definiciones}
%\theoremstyle{remark}
\newtheorem*{myobs}{Observación}
\newtheorem*{remark}{Importante}

\renewcommand{\baselinestretch}{1}  %interlineado

\addto\captionsspanish{%
  \renewcommand{\figurename}{Figura}%
}

\newcommand\myText[1]{\text{\scriptsize\tabular[t]{@{}l@{}}#1\endtabular}}
\addto\captionsspanish{%
  \renewcommand{\tablename}{Tabla}%
}

\def \ds {\displaystyle} %define un comando abreviado  
\def\com{“R”}

\usepackage{hyperref}%para referencias de internet con link!
\newcommand*{\fullref}[1]{\hyperref[{#1}]{ \nameref*{#1}}}
%comando \fullref para que ademas del número de capitulo, sección etc. escriba el título del capitulo, sección o lo que sea a lo que estamos haciendo referencia

\newcommand\comentario[1]{\textcolor{red}{#1}}%comentarios en el pdf

\interfootnotelinepenalty=10000 %previene que se pasen a otra página las notas de pie
\raggedbottom 
\addtolength{\topskip}{0pt plus 10pt}
\addtolength{\footnotesep}{0.1mm}

\VerbatimFootnotes%para poder usar Verbatim en las notas de pie

\begin{document}

\begin{enumerate}

\item  
%q
Calcular $\int_{0}^{\pi}x^2.\sin{x} dx$ sabiendo que $ \int_{0}^{\pi}x.\cos x dx = -2$

\item  
%q
Calcular $\int_{0}^{1}x^{10}. e^x dx$ sabiendo que $ \int_{0}^{1}x^9. e^x  = A$

\item  
%q
Si$ \int_{0}^{1}x^{20}. e^x dx =K$ enconces $ \int_{0}^{1}x^{21}. e^x$ es:
\begin{enumerate}
\item  $e - 21K$
\item  $e + 21K$
\item  $e - \frac{1}{21}K$
\item  $e + \frac{1}{21}K$
\end{enumerate}
Decir cuál es la opción correcta y justificar.
\item  
%q
Calcular: 
\begin{enumerate}
\item $\int_{0}^{1}x. e^{2x^2} dx$ 
\item $\int x. e^{-3x} dx$ 
\end{enumerate}

\item  
%q
Si$\int_{1}^{4} \frac{dx}{x\sqrt{x}}$ es:
\begin{enumerate}
\item  $\ln 4 + \frac{14}{3}$
\item  $2$
\item  $1$
\item  $\frac{3}{2}$
\end{enumerate}

\item  
%q
Calcular: 
\begin{enumerate}
\item $\int_{0}^{1}x. e^{2x^2} dx$ 
\item $\int x. \sin{2x}dx$ 
\end{enumerate}
 
\item  
%q
Calcular el área de la región encerrada entre $f(x) = x.e^{x^2}$ y $g(x) = x.e^{x+2}$. \textit{Pista}: buscar las primitivas de $f(x)$ y $g(x)$, luego los puntos de intersección de ambas funciones, plantear cómo se calcula el área y luego calcular usando las primitivas halladas.

\item  
%q
Si en la integral
\begin{equation*}
I =\int \frac{1}{1+e^{-2x}}dx
\end{equation*} 
se hace la sustitución $u = 1+e^{-2x}$, la integral que queda para calcular es:
\begin{enumerate}
\item $\frac{1}{2}\int \frac{1}{u.(u-1}du$ 
\item $-\frac{1}{2}\int \frac{1}{u.(u-1}du$ 
\item $2\int \frac{1}{u}du$
\item $-2\int \frac{1}{u}du$
 \end{enumerate}
Decir cuál es la opción correcta y calcular.

\item  
%q
Si en la integral
\begin{equation*}
I =\int \frac{1}{x\ln^5x}dx
\end{equation*} 
se hace la sustitución $u = \ln x$, la integral que queda para calcular es:
\begin{enumerate}
\item $\frac{1}{5}\int \frac{1}{e^u.u^5}du$ 
\item $-\frac{1}{5}\int \frac{1}{u^5}du$ 
\item $\int \frac{1}{u^5}du$
\item $\int \frac{1}{e^u.u^5}du$
 \end{enumerate}
Decir cuál es la opción correcta y calcular.

\item  
%q
Si en la integral
\begin{equation*}
\int_{0}^{1} \cos{\sqrt{x}}dx
\end{equation*} 
se hace la sustitución $u = \sqrt{x}$, la integral que queda para calcular es:
\begin{enumerate}
\item $2\int_{0}^{1} u\cos u du$ 
\item $\frac{1}{2}\int_{0}^{1} \frac{\cos u }{u} du$
\item $\frac{1}{2}\int_{0}^{1} u\cos u du$
\item $2 \int_{0}^{1} \frac{\cos u }{u} $
 \end{enumerate}
Decir cuál es la opción correcta y calcular.

%q
\end{enumerate}

\end{document}
