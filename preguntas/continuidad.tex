%no borrar PREAMBULO
\documentclass[12pt]{article}

\usepackage[top=3.5 cm, bottom=2.5  cm, left=3 cm, right=3 cm]{geometry}
\usepackage{fancyhdr}
\pagestyle{fancy}

\usepackage[hidelinks]{hyperref} %esta opción saca las cajas de colores de los hiperlinks

\fancyfoot[C]{\thepage }  %numera las páginas

\usepackage[utf8]{inputenc}

\usepackage{amsmath,amsfonts,amssymb}
\usepackage{xcolor}
\usepackage{fancyvrb}
\newcommand\verbbf[1]{\textcolor[rgb]{0,0,1}}%comando para colorear el texto en verbatim

%\linespread{1} %por si queremos achicar el espacio entre lineas

\usepackage{tabularx,booktabs}
\usepackage{graphicx}
\usepackage{float} %para que las figuras puedan ponerse en cualquier lado

\usepackage{subcaption}
\usepackage{layout}
\usepackage{multicol}  %para escribir en columnnas 
\usepackage{float}
\usepackage{textcomp}
\usepackage{natbib}
\usepackage{tikz}
\usepackage{multirow} %para cambiar el alto de una fila en una tabla
\tikzset{
  connect/.style = { dashed, gray }
}
\usepackage{pgfplots}
\pgfplotsset{compat=1.8}
\usepackage[english ,spanish]{babel}
\usepackage{latexsym}
\usepackage{verbatim}

%\usepackage{alltt}
\usepackage{indentfirst}

\usepackage{fancybox, calc} 

\usepackage[flushmargin]{footmisc} %para alinear las notas de página

\usepackage{url}
\usepackage{advdate}
\usepackage{wrapfig}
\usepackage{amsthm}
\usepackage[inline]{enumitem} %para hacer listas en una linea, los mismos comandos con *
\newtheorem*{myteo}{Teorema} % la * es para no numerarlos
\newtheorem*{myexample}{Ejemplo}
\newtheorem*{myprop}{Proposición}
\newtheorem*{mylem}{Lema}
\theoremstyle{definition}
\newtheorem*{mydef}{Definición}
\newtheorem{ejer}{Ejercicio}
\newtheorem*{mydefs}{Definiciones}
%\theoremstyle{remark}
\newtheorem*{myobs}{Observación}
\newtheorem*{remark}{Importante}

\renewcommand{\baselinestretch}{1}  %interlineado

\addto\captionsspanish{%
  \renewcommand{\figurename}{Figura}%
}

\newcommand\myText[1]{\text{\scriptsize\tabular[t]{@{}l@{}}#1\endtabular}}
\addto\captionsspanish{%
  \renewcommand{\tablename}{Tabla}%
}

\def \ds {\displaystyle} %define un comando abreviado  
\def\com{“R”}

\usepackage{hyperref}%para referencias de internet con link!
\newcommand*{\fullref}[1]{\hyperref[{#1}]{ \nameref*{#1}}}
%comando \fullref para que ademas del número de capitulo, sección etc. escriba el título del capitulo, sección o lo que sea a lo que estamos haciendo referencia

\newcommand\comentario[1]{\textcolor{red}{#1}}%comentarios en el pdf

\interfootnotelinepenalty=10000 %previene que se pasen a otra página las notas de pie
\raggedbottom 
\addtolength{\topskip}{0pt plus 10pt}
\addtolength{\footnotesep}{0.1mm}

\VerbatimFootnotes%para poder usar Verbatim en las notas de pie

\begin{document}

\begin{enumerate}

\item  
%q
Dada la función definida por:
\begin{equation*}
f(x) = 
\begin{cases} 
 \frac{1}{x^2}  & \text{si      } x < 0 \\
x^2-3x-2& \text{si  } 0\leq x < 3\\
(\frac{2x}{3} - 1)^2  & \text{si     } x \geq 3
\end{cases}
\end{equation*}

Se pide:

 \begin{enumerate}
        \item Graficar
        \item Calcular:
\begin{itemize}
\item  $\lim\limits_{x \to 0^+} f(x) $
\item  $\lim\limits_{x \to 0^-} f(x) $
\item  $\lim\limits_{x \to 3^+} f(x) $
\item  $\lim\limits_{x \to 3^-} f(x) $
\end{itemize}
        \item Decir si $x = 0$ y $x = 3$ son puntos de discontinuidad, y en tal caso, de qué tipo de discontinuidad de trata y por qué.
        \item Si es posible, redefinir la función (cambiarla \textit{a piacere}) en el intervalo $(-\infty, 0)$ para que la función resulte continua para todo  $x \in \mathbb{R}$ 
        \item Para la función f(x) (la continua o la otra, es irrelevante) calcular:
\begin{itemize}
\item  $\lim\limits_{x \to +\infty} f(x) $
\item  $\lim\limits_{x \to -\infty} f(x) $
\end{itemize}
\end{enumerate}

\item  
%q
Dada la función definida por:
\begin{equation*}
	f(x) = 
	\begin{cases} 
		\frac{x^2-x}{-x^2+4x-3}  & \text{si  } x <1 \\
		................& \text{s i } 1\leq x \leq  2\\
		\frac{x^2-2x}{-x^2+x+2}  & \text{si  } x > 2
	\end{cases}
\end{equation*}

Se pide:

\begin{enumerate}
	\item Calcular:
		\begin{itemize}
			\item  $\lim\limits_{x \to 1^-} f(x) $
			\item  $\lim\limits_{x \to 2^+} f(x) $
		\end{itemize}
		\item Si es posible, completar la función en el intervalo $[1,2]$ de modo que la función resulte continua en $x =1$ y $x =2$. Justificar. 
	\item Graficar la función según se completó en el ítem anterior.
	\item Si es posible, completar la función en el intervalo $[1,2]$ de modo que la función resulte continua en $x =1$ , pero discontinua en $x =2$ . Justificar la elección y decir qué tipo de discontinuidad se tiene en $x =1$ y por qué.
	\item Graficar la función según se completó en el ítem anterior.
	\item Para la función f(x) (la continua o la otra, es irrelevante) calcular
		\begin{itemize}
			\item  $\lim\limits_{x \to +\infty} f(x) $
			\item  $\lim\limits_{x \to -\infty} f(x) $
		\end{itemize}
\end{enumerate}

\item  
%q
Dada la función definida por:
\begin{equation*}
	f(x) = 
	\begin{cases} 
		\frac{x^3-x^2}{x^2+2x}  & \text{si  } x < 0 \\
		..................& \text{si   } 0 \leq x \leq 1\\
		\frac{1}{2x^2+4x-6}  & \text{si  } x > 1
	\end{cases}
\end{equation*}

Se pide:
\begin{enumerate}
	\item Calcular:
		\begin{itemize}
			\item  $\lim\limits_{x \to 0^-} f(x) $
			\item  $\lim\limits_{x \to 1^+} f(x) $
			\end{itemize}
	\item Si es posible, completar  la función en el intervalo $[0,1]$ de modo que la función resulte continua en $x = 0$ y $x =1$. Justificar. 
	\item Graficar la función según se completó en el ítem anterior.
	\item Si es posible, completar  la función en el intervalo $[0,1]$ de modo que la función resulte continua en $x = 0$ , pero discontinua en $x =1$ . Justificar la elección y decir qué tipo de discontinuidad se tiene en $x =1$ y por qué.
	\item Graficar la función según se completó en el ítem anterior.
\end{enumerate}

\item  
%q
Dada la función definida por:
\begin{equation*}
	f(x) = 
	\begin{cases} 
		\frac{(x-1)^2}{x^2-1 }  & \text{si  } x < 1 \\
		..................& \text{si   } 0 \leq x \leq 1\\
		\frac{1}{x^2-4x+3}  & \text{si  } x > 1
	\end{cases}
\end{equation*}

Se pide:
\begin{enumerate}
	\item Calcular:
		\begin{itemize}
			\item  $\lim\limits_{x \to 0^-} f(x) $
			\item  $\lim\limits_{x \to 1^+} f(x) $
			\end{itemize}
	\item Completar la función en el intervalo $[0,1]$ de modo que la función resulte continua en $x = 0$ y $x =1$. Justificar la elección. 
	\item Graficar la función según se completó en el ítem anterior.
	\item Completar la función en el intervalo $[0,1]$ de modo que la función resulte continua en $x = 0$ , pero discontinua en $x =1$ . Justificar la elección y decir qué tipo de discontinuidad se tiene en $x =1$ y por qué.
	\item Graficar la función según se completó en el ítem anterior.
\end{enumerate}
%q
\end{enumerate}

\end{document}
